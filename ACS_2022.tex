% Options for packages loaded elsewhere
\PassOptionsToPackage{unicode}{hyperref}
\PassOptionsToPackage{hyphens}{url}
\PassOptionsToPackage{dvipsnames,svgnames,x11names}{xcolor}
%
\documentclass[
  letterpaper,
  DIV=11,
  numbers=noendperiod]{scrartcl}

\usepackage{amsmath,amssymb}
\usepackage{iftex}
\ifPDFTeX
  \usepackage[T1]{fontenc}
  \usepackage[utf8]{inputenc}
  \usepackage{textcomp} % provide euro and other symbols
\else % if luatex or xetex
  \usepackage{unicode-math}
  \defaultfontfeatures{Scale=MatchLowercase}
  \defaultfontfeatures[\rmfamily]{Ligatures=TeX,Scale=1}
\fi
\usepackage{lmodern}
\ifPDFTeX\else  
    % xetex/luatex font selection
\fi
% Use upquote if available, for straight quotes in verbatim environments
\IfFileExists{upquote.sty}{\usepackage{upquote}}{}
\IfFileExists{microtype.sty}{% use microtype if available
  \usepackage[]{microtype}
  \UseMicrotypeSet[protrusion]{basicmath} % disable protrusion for tt fonts
}{}
\makeatletter
\@ifundefined{KOMAClassName}{% if non-KOMA class
  \IfFileExists{parskip.sty}{%
    \usepackage{parskip}
  }{% else
    \setlength{\parindent}{0pt}
    \setlength{\parskip}{6pt plus 2pt minus 1pt}}
}{% if KOMA class
  \KOMAoptions{parskip=half}}
\makeatother
\usepackage{xcolor}
\setlength{\emergencystretch}{3em} % prevent overfull lines
\setcounter{secnumdepth}{-\maxdimen} % remove section numbering
% Make \paragraph and \subparagraph free-standing
\makeatletter
\ifx\paragraph\undefined\else
  \let\oldparagraph\paragraph
  \renewcommand{\paragraph}{
    \@ifstar
      \xxxParagraphStar
      \xxxParagraphNoStar
  }
  \newcommand{\xxxParagraphStar}[1]{\oldparagraph*{#1}\mbox{}}
  \newcommand{\xxxParagraphNoStar}[1]{\oldparagraph{#1}\mbox{}}
\fi
\ifx\subparagraph\undefined\else
  \let\oldsubparagraph\subparagraph
  \renewcommand{\subparagraph}{
    \@ifstar
      \xxxSubParagraphStar
      \xxxSubParagraphNoStar
  }
  \newcommand{\xxxSubParagraphStar}[1]{\oldsubparagraph*{#1}\mbox{}}
  \newcommand{\xxxSubParagraphNoStar}[1]{\oldsubparagraph{#1}\mbox{}}
\fi
\makeatother

\usepackage{color}
\usepackage{fancyvrb}
\newcommand{\VerbBar}{|}
\newcommand{\VERB}{\Verb[commandchars=\\\{\}]}
\DefineVerbatimEnvironment{Highlighting}{Verbatim}{commandchars=\\\{\}}
% Add ',fontsize=\small' for more characters per line
\usepackage{framed}
\definecolor{shadecolor}{RGB}{241,243,245}
\newenvironment{Shaded}{\begin{snugshade}}{\end{snugshade}}
\newcommand{\AlertTok}[1]{\textcolor[rgb]{0.68,0.00,0.00}{#1}}
\newcommand{\AnnotationTok}[1]{\textcolor[rgb]{0.37,0.37,0.37}{#1}}
\newcommand{\AttributeTok}[1]{\textcolor[rgb]{0.40,0.45,0.13}{#1}}
\newcommand{\BaseNTok}[1]{\textcolor[rgb]{0.68,0.00,0.00}{#1}}
\newcommand{\BuiltInTok}[1]{\textcolor[rgb]{0.00,0.23,0.31}{#1}}
\newcommand{\CharTok}[1]{\textcolor[rgb]{0.13,0.47,0.30}{#1}}
\newcommand{\CommentTok}[1]{\textcolor[rgb]{0.37,0.37,0.37}{#1}}
\newcommand{\CommentVarTok}[1]{\textcolor[rgb]{0.37,0.37,0.37}{\textit{#1}}}
\newcommand{\ConstantTok}[1]{\textcolor[rgb]{0.56,0.35,0.01}{#1}}
\newcommand{\ControlFlowTok}[1]{\textcolor[rgb]{0.00,0.23,0.31}{\textbf{#1}}}
\newcommand{\DataTypeTok}[1]{\textcolor[rgb]{0.68,0.00,0.00}{#1}}
\newcommand{\DecValTok}[1]{\textcolor[rgb]{0.68,0.00,0.00}{#1}}
\newcommand{\DocumentationTok}[1]{\textcolor[rgb]{0.37,0.37,0.37}{\textit{#1}}}
\newcommand{\ErrorTok}[1]{\textcolor[rgb]{0.68,0.00,0.00}{#1}}
\newcommand{\ExtensionTok}[1]{\textcolor[rgb]{0.00,0.23,0.31}{#1}}
\newcommand{\FloatTok}[1]{\textcolor[rgb]{0.68,0.00,0.00}{#1}}
\newcommand{\FunctionTok}[1]{\textcolor[rgb]{0.28,0.35,0.67}{#1}}
\newcommand{\ImportTok}[1]{\textcolor[rgb]{0.00,0.46,0.62}{#1}}
\newcommand{\InformationTok}[1]{\textcolor[rgb]{0.37,0.37,0.37}{#1}}
\newcommand{\KeywordTok}[1]{\textcolor[rgb]{0.00,0.23,0.31}{\textbf{#1}}}
\newcommand{\NormalTok}[1]{\textcolor[rgb]{0.00,0.23,0.31}{#1}}
\newcommand{\OperatorTok}[1]{\textcolor[rgb]{0.37,0.37,0.37}{#1}}
\newcommand{\OtherTok}[1]{\textcolor[rgb]{0.00,0.23,0.31}{#1}}
\newcommand{\PreprocessorTok}[1]{\textcolor[rgb]{0.68,0.00,0.00}{#1}}
\newcommand{\RegionMarkerTok}[1]{\textcolor[rgb]{0.00,0.23,0.31}{#1}}
\newcommand{\SpecialCharTok}[1]{\textcolor[rgb]{0.37,0.37,0.37}{#1}}
\newcommand{\SpecialStringTok}[1]{\textcolor[rgb]{0.13,0.47,0.30}{#1}}
\newcommand{\StringTok}[1]{\textcolor[rgb]{0.13,0.47,0.30}{#1}}
\newcommand{\VariableTok}[1]{\textcolor[rgb]{0.07,0.07,0.07}{#1}}
\newcommand{\VerbatimStringTok}[1]{\textcolor[rgb]{0.13,0.47,0.30}{#1}}
\newcommand{\WarningTok}[1]{\textcolor[rgb]{0.37,0.37,0.37}{\textit{#1}}}

\providecommand{\tightlist}{%
  \setlength{\itemsep}{0pt}\setlength{\parskip}{0pt}}\usepackage{longtable,booktabs,array}
\usepackage{calc} % for calculating minipage widths
% Correct order of tables after \paragraph or \subparagraph
\usepackage{etoolbox}
\makeatletter
\patchcmd\longtable{\par}{\if@noskipsec\mbox{}\fi\par}{}{}
\makeatother
% Allow footnotes in longtable head/foot
\IfFileExists{footnotehyper.sty}{\usepackage{footnotehyper}}{\usepackage{footnote}}
\makesavenoteenv{longtable}
\usepackage{graphicx}
\makeatletter
\def\maxwidth{\ifdim\Gin@nat@width>\linewidth\linewidth\else\Gin@nat@width\fi}
\def\maxheight{\ifdim\Gin@nat@height>\textheight\textheight\else\Gin@nat@height\fi}
\makeatother
% Scale images if necessary, so that they will not overflow the page
% margins by default, and it is still possible to overwrite the defaults
% using explicit options in \includegraphics[width, height, ...]{}
\setkeys{Gin}{width=\maxwidth,height=\maxheight,keepaspectratio}
% Set default figure placement to htbp
\makeatletter
\def\fps@figure{htbp}
\makeatother

\KOMAoption{captions}{tableheading}
\makeatletter
\@ifpackageloaded{caption}{}{\usepackage{caption}}
\AtBeginDocument{%
\ifdefined\contentsname
  \renewcommand*\contentsname{Table of contents}
\else
  \newcommand\contentsname{Table of contents}
\fi
\ifdefined\listfigurename
  \renewcommand*\listfigurename{List of Figures}
\else
  \newcommand\listfigurename{List of Figures}
\fi
\ifdefined\listtablename
  \renewcommand*\listtablename{List of Tables}
\else
  \newcommand\listtablename{List of Tables}
\fi
\ifdefined\figurename
  \renewcommand*\figurename{Figure}
\else
  \newcommand\figurename{Figure}
\fi
\ifdefined\tablename
  \renewcommand*\tablename{Table}
\else
  \newcommand\tablename{Table}
\fi
}
\@ifpackageloaded{float}{}{\usepackage{float}}
\floatstyle{ruled}
\@ifundefined{c@chapter}{\newfloat{codelisting}{h}{lop}}{\newfloat{codelisting}{h}{lop}[chapter]}
\floatname{codelisting}{Listing}
\newcommand*\listoflistings{\listof{codelisting}{List of Listings}}
\makeatother
\makeatletter
\makeatother
\makeatletter
\@ifpackageloaded{caption}{}{\usepackage{caption}}
\@ifpackageloaded{subcaption}{}{\usepackage{subcaption}}
\makeatother

\ifLuaTeX
  \usepackage{selnolig}  % disable illegal ligatures
\fi
\usepackage{bookmark}

\IfFileExists{xurl.sty}{\usepackage{xurl}}{} % add URL line breaks if available
\urlstyle{same} % disable monospaced font for URLs
\hypersetup{
  pdftitle={IPUMS ACS 2022: Ratio Estimators and State Respondent Counts},
  pdfauthor={Doran Wang},
  colorlinks=true,
  linkcolor={blue},
  filecolor={Maroon},
  citecolor={Blue},
  urlcolor={Blue},
  pdfcreator={LaTeX via pandoc}}


\title{IPUMS ACS 2022: Ratio Estimators and State Respondent Counts}
\author{Doran Wang}
\date{}

\begin{document}
\maketitle

\renewcommand*\contentsname{Table of contents}
{
\hypersetup{linkcolor=}
\setcounter{tocdepth}{2}
\tableofcontents
}

\section{Introduction}\label{introduction}

In this report, we will estimate the total number of respondents in each
U.S. state using the ratio estimators approach of Laplace. Our focus is
on respondents with doctoral degrees as their highest educational
attainment, based on the 2022 American Community Survey (ACS) data
obtained from IPUMS.

\section{Data Acquisition}\label{data-acquisition}

To obtain the data, follow these steps:

\begin{enumerate}
\def\labelenumi{\arabic{enumi}.}
\item
  Visit the \href{https://usa.ipums.org/usa/}{IPUMS USA} website.
\item
  Create a free account and log in.
\item
  Select the ``2022 ACS'' dataset.
\item
  Choose the following variables:

  \begin{itemize}
  \tightlist
  \item
    \textbf{STATEICP}: State code
  \item
    \textbf{EDUCD}: Detailed educational attainment
  \end{itemize}
\item
  Submit your data extract request and download the resulting
  \texttt{.csv} file. If the file is compressed (with a \texttt{.gz}
  extension), use the following command in your terminal to unzip it:

  ```bash gunzip usa\_00004.csv.gz
\item
  Load the data into R
\end{enumerate}

\begin{center}\rule{0.5\linewidth}{0.5pt}\end{center}

\subsection{Overview of the Ratio Estimator
Approach}\label{overview-of-the-ratio-estimator-approach}

The ratio estimator approach is a simple method used to estimate
population totals based on a known ratio from a subset of data. In this
case, we know the number of respondents with doctoral degrees in each
state and the total number of respondents in California (391,171). We
can compute the ratio of doctoral degree holders to total respondents in
California and apply that ratio to estimate total respondent counts in
other states.

Mathematically, the ratio estimator is given by:

\(N=\frac{D_{D\_{CA}}×{N\_{CA}N^j\)

Where:

\begin{itemize}
\item
  \$\textbackslash hat\{N\}\_j\$ is the estimated total number of
  respondents in state \$j\$.
\item
  \$D\_j\$ is the number of respondents with doctoral degrees in state
  \$j\$.
\item
  \$D\_\{CA\}\$ is the number of respondents with doctoral degrees in
  California.
\item
  \$N\_\{CA\}\$ is the total number of respondents in California
  (391,171).
\end{itemize}

\section{Data Analysis}\label{data-analysis}

\begin{Shaded}
\begin{Highlighting}[]
\CommentTok{\# Load necessary libraries}
\FunctionTok{library}\NormalTok{(tidyverse)}
\end{Highlighting}
\end{Shaded}

\begin{verbatim}
-- Attaching core tidyverse packages ------------------------ tidyverse 2.0.0 --
v dplyr     1.1.4     v readr     2.1.5
v forcats   1.0.0     v stringr   1.5.1
v ggplot2   3.5.1     v tibble    3.2.1
v lubridate 1.9.3     v tidyr     1.3.1
v purrr     1.0.2     
-- Conflicts ------------------------------------------ tidyverse_conflicts() --
x dplyr::filter() masks stats::filter()
x dplyr::lag()    masks stats::lag()
i Use the conflicted package (<http://conflicted.r-lib.org/>) to force all conflicts to become errors
\end{verbatim}

\begin{Shaded}
\begin{Highlighting}[]
\CommentTok{\# Load the IPUMS dataset (adjust the path to your file)}
\NormalTok{data }\OtherTok{\textless{}{-}} \FunctionTok{read\_csv}\NormalTok{(}\StringTok{"usa\_00001.csv"}\NormalTok{)}
\end{Highlighting}
\end{Shaded}

\begin{verbatim}
Rows: 3373378 Columns: 13
-- Column specification --------------------------------------------------------
Delimiter: ","
dbl (13): YEAR, SAMPLE, SERIAL, CBSERIAL, HHWT, CLUSTER, STATEICP, STRATA, G...

i Use `spec()` to retrieve the full column specification for this data.
i Specify the column types or set `show_col_types = FALSE` to quiet this message.
\end{verbatim}

\begin{Shaded}
\begin{Highlighting}[]
\CommentTok{\# Filter for respondents with doctoral degrees}
\NormalTok{doctoral\_data }\OtherTok{\textless{}{-}}\NormalTok{ data }\SpecialCharTok{\%\textgreater{}\%}
  \FunctionTok{filter}\NormalTok{(EDUCD }\SpecialCharTok{==} \DecValTok{116}\NormalTok{)  }\CommentTok{\# 116 is the code for doctoral degree}

\CommentTok{\# Count the number of doctoral degree holders by state}
\NormalTok{doctoral\_by\_state }\OtherTok{\textless{}{-}}\NormalTok{ doctoral\_data }\SpecialCharTok{\%\textgreater{}\%}
  \FunctionTok{group\_by}\NormalTok{(STATEICP) }\SpecialCharTok{\%\textgreater{}\%}
  \FunctionTok{summarise}\NormalTok{(}\AttributeTok{doctoral\_count =} \FunctionTok{n}\NormalTok{())}

\CommentTok{\# Example: Number of doctoral degree holders in California}
\NormalTok{california\_doctoral }\OtherTok{\textless{}{-}}\NormalTok{ doctoral\_by\_state }\SpecialCharTok{\%\textgreater{}\%}
  \FunctionTok{filter}\NormalTok{(STATEICP }\SpecialCharTok{==} \DecValTok{6}\NormalTok{)  }\CommentTok{\# 06 is the code for California}

\NormalTok{california\_doctoral\_count }\OtherTok{\textless{}{-}}\NormalTok{ california\_doctoral}\SpecialCharTok{$}\NormalTok{doctoral\_count}

\CommentTok{\# Compute the ratio and apply it to estimate total respondents by state}
\NormalTok{ratio }\OtherTok{\textless{}{-}}\NormalTok{ doctoral\_by\_state }\SpecialCharTok{\%\textgreater{}\%}
  \FunctionTok{mutate}\NormalTok{(}
    \AttributeTok{estimated\_total =}\NormalTok{ (doctoral\_count }\SpecialCharTok{/}\NormalTok{ california\_doctoral\_count) }\SpecialCharTok{*} \DecValTok{391171}
\NormalTok{  )}

\CommentTok{\# Display the results}
\FunctionTok{print}\NormalTok{(ratio)}
\end{Highlighting}
\end{Shaded}

\begin{verbatim}
# A tibble: 51 x 3
   STATEICP doctoral_count estimated_total
      <dbl>          <int>           <dbl>
 1        1            600        1791623.
 2        2            165         492696.
 3        3           2014        6013881.
 4        4            244         728593.
 5        5            177         528529.
 6        6            131         391171 
 7       11            152         453878.
 8       12           1438        4293923.
 9       13           2829        8447502.
10       14           1620        4837382.
# i 41 more rows
\end{verbatim}

\subsection{Running Code}\label{running-code}

When you click the \textbf{Render} button a document will be generated
that includes both content and the output of embedded code. You can
embed code like thi




\end{document}
